\newcommand{\PhoneLab}{\textsc{PhoneLab}}

% 16 Jan 2011 : GWA : Tagline.

Currently I am a post-doctoral scholar supervised by Professor Hari
Balakrishnan at MIT. In August, 2011, I will begin a junior faculty position
at the University at Buffalo.

% 16 Jan 2011 : Research agenda.

\uline{I am interested in power as a design constraint spanning all classes
of computer systems}---embedded sensors, mobile phones, personal computers,
datacenter servers, and large-scale networks. My research agenda is rooted in
the belief that studying power consumption and management at multiple levels
of the computing infrastructure will yield insights relevant \textit{at all
levels}. Power consumption is a major barrier to the advance of pervasive
computing, and at {\scshape PC3} I will contribute my interest and expertise
in this area.

% 16 Jan 2011 : GWA : Prior work.

My graduate research completed earlier this year focused on energy management
for wireless sensor networks. I built three systems---Lance, IDEA and
Peloton---that manage energy at the \textit{network}, rather than node,
level. Lance improves the performance of data-intensive sensor network
applications by considering both the cost and value of information when
collecting data. IDEA provides a network-wide energy coordination layer
facilitating energy optimizations impossible for a single node to perform
alone. Peloton proposes a distributed operating system for coordinated
resource management built on state sharing, a distributed energy ticket
abstraction, and local neighborhood ticket management. Along similar lines, I
contributed to the development of PowerTOSSIM, which added power modeling and
to the TinyOS simulator; the Pixie sensor node operating system, which
promoted energy to a first-class system resource; and MoteLab, a public
testbed supporting all kinds of sensor network research.

% 16 Jan 2011 : GWA : MIT work.

A present I am expanding the interest in power adaptation I developed while
working on sensor networks in three directions: wireless networking,
operating system design, and mobile computing. All these areas are central to
pervasive computing, and my overall goal is to reduce, manage, and understand
the energy consumption of pervasive devices.

\textbf{{\scshape Wireless Networking:}} At MIT I am working with Prof.
Balakrishnan on techniques to save power in Wifi networks. We are
experimenting with physical-layer techniques enabling rapid packet source and
destination detection that, if accurate, could allow clients to discard
packets not addressed to them without enabling power-hungry digital decoding.
We are leveraging the power of the AirBlue FPGA-based 802.11 software radio,
which unites a stack that is easily to modify and hardware capable of
real-time processing. AirBlue is almost capable of interoperating with
commodity devices and providing the platform for real
wireless protocol experimentation the community currently lacks.

% 16 Jan 2011 : GWA : Power-agile computing.

\textbf{{\scshape Operating System Design:}} With Mark Hempstead at Drexel
University I am designing operating systems capable of powering the next
generation of heterogeneous power-proportional hardware architectures which
incorporate multiple components with different power-performance tradeoffs.
We have coined the term \textit{power agility} to describe the ability of a
system to operate these devices balancing performance and power consumption.
Given increasingly heterogeneous devices, power agility requires not merely
adjusting individual components but activating and deactivating them to react
to changes in demand caused by variations in device usage. On power-agile
devices scheduling and resource allocation are complicated by the fluid
nature of the underlying hardware. We are currently addressing the challenges
inherent to the five roles that the operating system plays while operating
power-agile hardware: measuring and predicting performance; along with
selecting, preparing and executing device state transitions.

% 16 Jan 2011 : GWA : PhoneLab.

\textbf{{\scshape Mobile Computing:}} Together with SUNY Buffalo faculty
Murat Demirbas, Steve Ko and Tevfik Kosar I am building a large-scale
participatory smartphone testbed. Having previously built MoteLab, I know
first-hand how important testbeds are to advancing research, and no public
smartphone testbeds exists today. We envision a testbed called \PhoneLab{}
that enables smartphone operating system and mobile application research in a
realistic environment at a scale not previously possible. \PhoneLab{} will
consist of a large number---1,000 or more---of reprogrammable Android devices
used by students and staff at SUNY Buffalo. \PhoneLab{} aims to provide
\textbf{power}, \textbf{scale}, and \textbf{realism}. Power, to allow the
modification of smartphone software above and below the OS-application
interface while simplifying instrumentation and data collection to facilitate
efficient experimentation. Scale, to provide access to an order of magnitude
more participants than typically used by smartphone studies. And realism, by
minimizing experimental disturbance and allowing participants to use their
smartphones naturally. I see \PhoneLab{} as a central part of future
experiments on improving the battery lifetimes and energy-efficiency of
smartphones, as well as using smartphones to facilitate interaction between
users and their environment.

% 16 Jan 2011 : GWA : More on working with Indian scientists on PhoneLab.

We are eager to collaborate with other institutions to develop similar
testbeds that can be federated with \PhoneLab{}. India has phone-savvy
citizens and students as well as excellent technical universities that would
be an ideal partners in this effort. The capablities of phones usage of
features are likely to vary across cultures, and it would be interesting to
see what different usage patterns globally-distributed testbeds would
uncover. We also foresee \PhoneLab{} as an experimental environment for
developing pervasive applications appropriate for use in newly-industrialized
countries like India. Public-health applications are one example, and a group
at SUNY Buffalo is already collaborating with public health researchers to
use smartphone sensing to estimate exposure to pollutants, a project we will
deploy on \PhoneLab{} when ready.

% 16 Jan 2011 : GWA : International collaborations in Ecuador and on MoteLab.
% 16 Jan 2011 : GWA : Personal contacts with Indian scientists.

I have previous experience with cross-cultural collaboration obtained during
the Harvard volcano-monitoring project which I helped lead. Working closely
with scientists at the Instituto Geof\'{i}sico at the Escuela Politecnica
Nacional (IGEPN) I participated in three field deployments on active
Ecuadorean volcanos over four years. I have also enjoyed professional and
personal relationships with several excellent Indian-American scientists:
Kiran Muriswamy-Reddy, a former graduate-school colleague, now at Amazon;
Karthik Dantu, currently a post-doc on the RoboBees project; and Rohan Murty,
a graduate-school colleague and close friend. I have found Indian scientists
enjoyable to work with, and am eager to initiate contacts with them early in
my professional career.
